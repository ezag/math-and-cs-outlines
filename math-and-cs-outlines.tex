\documentclass[a4paper]{book}
\usepackage[utf8]{inputenc}
\usepackage[T1,T2A]{fontenc}
\usepackage[russian]{babel}
\usepackage{csquotes}
\usepackage[hidelinks]{hyperref}
\usepackage{shorttoc}

\usepackage[backend=biber,sorting=none]{biblatex}
\addbibresource{bibliography/sources.bib}
\addbibresource{bibliography/coursera.bib}
\addbibresource{bibliography/lisovik.bib}

% Workaround - http://tex.stackexchange.com/a/277317
\NewBibliographyString{langjapanese}
\NewBibliographyString{fromjapanese}

\usepackage[type={CC},modifier={by-sa},version={3.0}]{doclicense}

\title{Конспекты по математике и информатике}
\author{Евгений Загородний}

\usepackage{indentfirst}

\begin{document}

\frontmatter
\maketitle
\pagestyle{empty}
\vspace*{\fill}
\doclicenseThis

\shorttoc{Содержание}{0}
\tableofcontents
\chapter{Введение}
\section{Для чего?}
\section{Обозначения}


\mainmatter
\chapter{Математическая логика}

Материалы для главы см. в книгах \cite{galiev2002}, \cite{kolmogorov2006}.

А также в курсе Coursera \cite{coursera:logic-introduction}.

Успехов!

\chapter{Теория множеств}

\chapter{Теория вероятностей}

\chapter{Математический анализ}

\chapter{Линейная алгебра}

\chapter{Криптография}

См. курс \cite{coursera-crypto}.

\chapter{Алгоритмы}

См. курс \cite{c:a-d-a}.
\section{Структуры данных}
\section{Сортировка}
\section{Поиск}
\section{Графы}

\chapter{Машинное обучение}
\section{Регрессия}
\subsection{Линейная регрессия}
\subsubsection{Линейная регрессия с одной переменной}
\subsubsection{Линейная регрессия c несколькими переменными}
\subsection{Логистическая регрессия}
\section{Классификация}
\section{Кластеризация}


\appendix
\chapter{Понятия, используемые без определения}

\begin{itemize}
  \item множество
  \item матрица
  \item функция
\end{itemize}

\chapter{Теоремы, используемые без доказательства}
\chapter{Теоремы, используемые без формулировки}

\backmatter
\printbibliography

\end{document}
